% \documentclass{ctexart}
\documentclass{article}
\usepackage[a4paper, margin=2cm]{geometry}
\usepackage{listings}
\usepackage{courier}
\usepackage{algorithm}
\usepackage{algpseudocode}
\usepackage{ctex}

\makeatletter
\setlength{\@fptop}{0pt}  
\makeatother

\algnewcommand{\IIf}[1]{\State\algorithmicif\ #1\ \algorithmicthen}
\algnewcommand{\EndIIf}{\unskip\ \algorithmicend\ \algorithmicif}
\algnewcommand{\FFor}[1]{\State\algorithmicfor\ #1\ \algorithmicdo}
\algnewcommand{\EndFFor}{\unskip\ \algorithmicend\ \algorithmicfor}
\algnewcommand\True{\textbf{true}\space}
\algnewcommand\False{\textbf{false}\space}
\algnewcommand\Continue{\textbf{continue}\space}

\lstset{
    basicstyle=\ttfamily
}

\setmainfont{Times New Roman}
\setCJKmainfont{TW-Kai-98_1}

\title{PA3: Detailed Routing for Photonic Integrated Circuits (PICs)}
\author{
  李品翰 \\
  四電機三甲 \\
  B11107051
}
\date{}  

\begin{document}

\maketitle

本次作業使用A*演算法完成,整體pseudocode如下:

\begin{algorithm}
\caption{Overall Code Structure}
\textbf{Input:} grid size, propagation loss, crossing loss, bending loss, input nets \\
\textbf{Output:} route of nets
\begin{algorithmic}

\State $grid \gets$ 2d integer array with all value set to 0 \Comment{作為地圖}
\State $nets \gets$ a empty set of net \Comment{儲存已繞好的線}

\For{$input\_net$ in $input\_nets$}
    \State $net\gets\textrm{L\_Routing}(intput\_net)$ \Comment{無視障礙物,以L繞線作為global routing}
    \State $nets \gets nets \cup \{net\}$
    \FFor{$point$ in $net$} grid[$point.x$][$point.y$]++ \EndFFor
\EndFor

\State $proceed \gets \True$
\While {$proceed == \True$} \Comment{存在重新繞過的net}
    \State $proceed \gets \False$
    \For{$net$ in $nets$}
        \State $nets \gets nets - \{net\}$ \Comment{移除舊net}
        \FFor{$point$ in $net$} grid[$point.x$][$point.y$]$--$ \EndFFor \Comment{在地圖上移除舊net}
        \State $new\_net \gets \textrm{A*\_Routing}(grid, net)$
        \If{$\textrm{loss}(grid, new\_net) < \textrm{loss}(grid, net)$}
            \State $nets \gets nets \cup \{new\_net\}$ \Comment{加入新net}
            \FFor{$point$ in $new\_net$} grid[$point.x$][$point.y$]++ \EndFFor \Comment{在地圖上加入新net}
            \State $proceed \gets \True$
        \Else
            \State $nets \gets nets \cup \{net\}$ \Comment{復原舊net}
            \FFor{$point$ in $net$} grid[$point.x$][$point.y$]++ \EndFFor \Comment{在地圖上復原舊net}
        \EndIf
    \EndFor
\EndWhile

\end{algorithmic}
\end{algorithm}

\begin{itemize}
    \item \texttt{grid}提供各net繞線時,得知某點是否有其他net存在,需要大量隨機存取,因此使用\texttt{std::vector}
    \item \texttt{nets}在L Routing後大小維持固定,若找到更佳繞線也可直接覆蓋,沒有特殊需求,因此使用最簡單的\texttt{std::vector}
\end{itemize}

\pagebreak

\begin{algorithm}
\caption{A* Routing Algorithm}
\textbf{Input:} grid, input net, propagation loss, crossing loss, bending loss \\
\textbf{Output:} route of net
\begin{algorithmic}

\State $points \gets \{\textrm{first\_point}(input\_net)\}$ \Comment{儲存已探索過的點}
\While{\True}
    \State $best\_point \gets \textrm{min\_total\_cost}(points)$
    \State lock($best\_point$)
    \If {$best\_point$ == $\textrm{last\_point}(input\_net)$} \Comment{找到終點}
        \State \Return net with points from \textrm{first\_point}($input\_net$) to \textrm{last\_point}($input\_net$) 
    \EndIf
    \For {$point$ in up, down, left, right of $best\_point$} \Comment{探索周圍點}
        \IIf {out\_of\_bound($point$)} \Continue \EndIIf
        \IIf {$point$ == privious\_point($best\_point$)} \Continue \EndIIf
        \State cost, estimate\_cost = evaluate\_cost($point, best\_point, grid, \textrm{last\_point}(input\_net)$)
        
        \If {$point \in points$}
            \State $old\_point \gets \textrm{find}(points, point)$
            \IIf {$\textrm{total\_cost}(point) \geq \textrm{total\_cost}(old\_point)$} \Continue \EndIIf \Comment{非更佳路線}
            \State $points \gets points - \{old\_point\}$ \Comment{新路徑較佳,刪除舊路徑}
        \EndIf

        \State $points \gets points \cup \{point\}$
    \EndFor
\EndWhile

\end{algorithmic}
\end{algorithm}


\begin{itemize}
    \item 探索到終點,準備回傳時,因路徑是由終點一路回推找到起點,需使用到\texttt{push\_front},因此\texttt{net}使用\texttt{std::deque}來儲存路徑上的點
    \item 為了在盡可能多的地方達成\texttt{O(1)}操作,\texttt{points}被我分成\texttt{id\_points}與\texttt{ordered\_points}兩個set: 
    \begin{itemize}
        \item \texttt{id\_points}: 使用\texttt{std::set},以點座標作為comparator,避免set中出現重複點
        \item \texttt{ordered\_points}: 使用\texttt{std::multiset},利用C++ set排序的特性,依序比較lock、total cost、estimate cost,此時只要呼叫\texttt{*ordered\_points.begin()}即可以\texttt{O(1)}時間取得total cost最小的點
    \end{itemize}
\end{itemize}

\end{document}
